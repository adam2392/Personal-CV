%-------------------------------------------------------------------------------
%	SECTION TITLE
%-------------------------------------------------------------------------------
\cvsection{Experience}


%-------------------------------------------------------------------------------
%	CONTENT
%-------------------------------------------------------------------------------
\cvsubsection{Research Experience}
\begin{cventries}
  
%---------------------------------------------------------
 % Usage: \cvsubentry{<position>}{<title>}{<date>}{<description>}
    \cvproject
        {Postdoctoral Research Scientist, Causal AI Lab, Columbia University \newline Advisor: Dr. Elias Bareinboim} % Institution
        {Jan. 2022 --- Jan. 2024} % Date(s)
        {
          \begin{cvitems} % Description(s) bullet points
            \item{Develop causal machine learning method for optimal adjustment in uncertain causal settings for estimating causal quantities.}
            \item{Develop a causal discovery method that discovers causal relationships and incorporates observational and interventional data from multiple domains}
            \item{Develop causal machine learning Python software in collaboration with Amazon, Microsoft and IBM researchers at Py-Why.}
          \end{cvitems}
        }
    \newline
    
    \cvproject
        {Graduate Research Assistant, Neuromedical Control Systems Lab, Johns Hopkins University \newline Advisor: Dr. Sridevi Sarma} % Institution
        {Aug. 2015 --- Dec. 2021} % Date(s)
        {
          \begin{cvitems} % Description(s) bullet points
            \item{Coordinated data pipeline of electrophysiological and clinical data of epilepsy patients from 5 hospitals in coordination with clinicians in setting up a HIPAA-compliant server for highly parallelized data analysis, resulting in \textbf{Nature Neuroscience publication}.}
            \item{Identified and developed signal processing and statistical analysis of clinical multi-modality datasets that resulted in over 400 pull requests merged in open-source packages with up to 1,000's of users (\textbf{Git, CI, unit-testing, software design \& development})}
            \item{Developed statistical and machine learning models on multivariate time series EEG, clinical and neuroimaging MRI and CT data to analyze different seizure localization models (model building \& validation with \textbf{scikit-learn/keras/pytorch}, data wrangling with \textbf{pandas,numpy}).}
            \item{Coordinated open-source discussions on EEG and iEEG data formatting in a 79 international team of researchers on Github (\textbf{technical communication of the Brain Imaging Data Structure - BIDS})}
            \item{Coordinated a team of engineers to develop a structure-aware Random Forest algorithm in Python and Cython to perform manifold learning (to be implemented as a PR into \textbf{scikit-learn}).}
          \end{cvitems}
        }
    \newline

    \cvproject
        {Visiting Research Scientist, Theoretical Neurosciences Group, Aix-Marseille University \newline Advisors: Dr. Viktor Jirsa, Dr. Sridevi Sarma} % Institution
        {Sep. 2017 --- Sep. 2018} % Location
        {
          \begin{cvitems} % Description(s) bullet points
            \item{Developed a high-throughput parallelized data pipeline for multi-modality 3D brain imaging using \textbf{Bash and Snakemake (Python DAG engine)} resulting in robust 3D brain visualizations.}
            \item{Designed \textbf{nonlinear biophysical simulation models} with \textbf{linear dynamical systems analysis} to predict the surgical outcome in epileptic patients resulting in a paper to be submitted to Brain}
            \item{Developed a supervised deep learning pipeline using nonlinear computational modeling and a Recurrent-CNN model to perform patient-specific seizure detection (\textbf{Python/Keras/Pytorch})}
            \item{Implemented open-source code on \href{https://github.com/the-virtual-brain/tvb-library/}{\textit{The Virtual Brain}} (a Human Brain Project) for generating observational noise, analysis of simulated source signals and scientific demos}
           % \item{Submitted manuscript detailing our findings}
          \end{cvitems}
        }
        
    % \newline
    % \cvproject
    %     {Undergraduate Researcher, Neural Interaction Lab, University of California - San Diego \newline Advisor: Dr. Todd Coleman} % Institution
    %     {Apr. 2013 --- Sep. 2015} % Date(s)
    %     {
    %       \begin{cvitems} % Description(s) bullet points
    %         \item{Researched and developed novel ways to evaluate Parkinson's disease using gait and 3D spatiotemporal data from the Microsoft Kinect in collaboration with Computer Vision Lab and School of Medicine.}
    %         \item{Developed data analytics software using C++ and Matlab for signal processing of coordinate time series data for the purpose of tracking biometrics of Parkinson’s disease patients}
    %         \item{Wrote a successful grant and IRB to carry out pilot clinical studies in collaboration with 3 professors; awarded the Gordon Fellowship Award for outstanding engineering leadership.}
    %         \item{Carried out validation and clinical experiments on 21 PD and 21 control subjects, while coordinating scheduling with clinicians and patients.}
    %       \end{cvitems}
    %     }
    % \newline
%---------------------------------------------------------
\end{cventries}

% \newpage

